\documentclass[final]{beamer}
\mode<presentation>

\usepackage{type1ec}
\usepackage[english,russian]{babel}

\usepackage[utf8x]{inputenc}

\usepackage{anyfontsize}
\usepackage[T1]{fontenc}
\usepackage{amsmath,amsthm, amssymb, latexsym}
\usepackage{calc}
\usepackage{exscale} % required to scale math fonts properly
\usepackage{cmbright}
\usepackage{ragged2e} 
\usepackage{svg}
\graphicspath{{images/}}

\usepackage{lipsum}  % used only to generate dummy text

%\usepackage[orientation=portrait,size=a0,scale=1.3]{beamerposter}
\usepackage[orientation=portrait,size=custom,width=65,height=100,scale=0.95]{beamerposter}

% use english or russian options to change text at footline
\usetheme[color, twocolumn, english]{LETI} 

%\setbeamertemplate{background}[grid]{} % use grid to aline block elements

%%%% Set up the title and author info
\titlestart{The looooooooooooooooooooooooooong poster title} 
\titleend{second line of title is optional}
\titlesize{70pt} % Use this to change title size if necessary

\author{Author One$^1$ and Author Two$^2$}
\institute{$^1$ Institution1, $^2$ Institution2}
\faculty{my faculty}
\department{my department}
\mail{my@email.ru}

% Stuff such as logos of contributing institutes can be put in the lower left corner using this
\leftcorner{}

\newcommand{\figfont}{\normalsize} % set fotsize for figures 

\begin{document}
\begin{poster}
%%%%%%%%%%%%%%%%%%%%%%%%%%%%%%%%%%%%%%%%%%%%%%%%%%%%%%%%%%%%%%%%%%%%%%%%%%%%%%
% First column %%%%%%%%%%%%%%%%%%%%%%%%%%%%%%%%%%%%%%%%%%%%%%%%%%%%%%%%%%%%%%%
\newcolumn

\section{Motivation} \justifying
\begin{itemize}
    \item First item
    \item second item
    \item Another item
\end{itemize}

\lipsum[2]

\section{First section}  \justifying
\lipsum[4]
 
\subsection{Subsection1 name}


$$
E = mc^2
$$

\subsection{Subsection2 name}
{\centering
\includesvg[width=\columnwidth]{images/fig}
\caption{Figure name}
\vspace{1ex}
\includesvg[width=0.6\columnwidth]{images/fig}
\vspace{2ex}
\caption{You can scale figures and text separately}

}
\vspace{1ex}
\lipsum[6]
%%%%%%%%%%%%%%%%%%%%%%%%%%%%%%%%%%%%%%%%%%%%%%%%%%%%%%%%%%%%%%%%%%%%%%%%%%%%%%
% Second column %%%%%%%%%%%%%%%%%%%%%%%%%%%%%%%%%%%%%%%%%%%%%%%%%%%%%%%%%%%%%%
\newcolumn

\section{Another Section} \justifying
\lipsum[3]

\vspace{2ex}
\newcommand{\figwidth}{0.6\columnwidth}

\subsection{Exhibit A}
\begin{columns}[c]
    \begin{column}{0.95\columnwidth-\figwidth}
        \begin{itemize}   \itemsep15pt    
            \item First item
            \item Second item
            \item item
            \item item
            \item item
            \item item
        \end{itemize}
        
    \end{column}
       
    \begin{column}{\figwidth}
        \centering
        \figfont
        \includegraphics[width=\columnwidth]{example-image-a}
        \vspace*{2ex}
        \caption{Figure name} 
    \end{column}
\end{columns}

\subsection{Exhibit B}
\begin{columns}[c]
	\begin{column}{0.95\columnwidth-\figwidth}
		\begin{itemize}   \itemsep15pt    
			\item First item
			\item Second item
		\end{itemize}
		
		$$
		E = mc^2
		$$
		
	\end{column}
	
	\begin{column}{\figwidth}
		\centering
		\figfont
		\includegraphics[width=\columnwidth]{example-image-b}
		\vspace*{2ex}
		\caption{Figure name} 
	\end{column}
\end{columns}



\lipsum[1-2]

\section{Summary} \justifying
\begin{itemize}
    \item First item
    \item second item
    \item Another item
\end{itemize}


\end{poster}
\end{document}