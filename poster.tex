\documentclass[final]{beamer}
\mode<presentation>

% STEP 1: Change the next line according to your language
\usepackage[english]{babel}

% STEP 2: Make sure this character encoding matches the one you save this file as
% (this template is utf8 by default but your editor may change it, causing problems)
\usepackage[utf8x]{inputenc}

% You probably don't need to touch the following four lines
\usepackage[T1]{fontenc}
\usepackage{lmodern}
\usepackage{amsmath,amsthm, amssymb, latexsym}
\usepackage{calc}
\usepackage{exscale} % required to scale math fonts properly
\usepackage{eulervm} % Euler VM symbols
\usepackage{ragged2e} 
\usepackage{svg}
\graphicspath{{images/}}

%\usepackage[orientation=portrait,size=a0,scale=1.3]{beamerposter}
\usepackage[orientation=portrait,size=custom,width=65,height=100,scale=0.95]{beamerposter}

% STEP 3:
% Change colours by setting \usetheme[<id>, twocolumn]{LETI}.
\usetheme[color, twocolumn]{LETI}

%\setbeamertemplate{background}[grid]{} % use grid to aline block elements

% STEP 4: Set up the title and author info
\titlestart{The looooooooooooooooooooooooooong poster title} 
\titleend{second line of title is optional}

%\titlesize{\Huge} % Use this to change title size if necessary

\author{Author One$^1$ and Author Two$^2$}
\institute{$^1$ Institution1, $^2$ Institution2}


% Stuff such as logos of contributing institutes can be put in the lower left corner using this
\leftcorner{}

\newcommand{\figfont}{\normalsize} % set fotsize for figures 

\begin{document}
\begin{poster}
%%%%%%%%%%%%%%%%%%%%%%%%%%%%%%%%%%%%%%%%%%%%%%%%%%%%%%%%%%%%%%%%%%%%%%%%%%%%%%
% First column %%%%%%%%%%%%%%%%%%%%%%%%%%%%%%%%%%%%%%%%%%%%%%%%%%%%%%%%%%%%%%%
\newcolumn

\section{Motivation} \justifying
\begin{itemize}
    \item First item
    \item second item
    \item Another item
\end{itemize}

Lorem ipsum dolor sit amet, consectetur adipiscing elit. Quisque blandit nisi at purus facilisis, consectetur maximus arcu mollis. Sed imperdiet nisi eros, sit amet sagittis enim sollicitudin at. Aenean metus massa, ornare eget placerat a, cursus convallis elit. Suspendisse potenti. Mauris ornare finibus leo, ut suscipit libero sagittis sit amet. 

\section{Frist section}
Cras sed nunc lacinia, tristique augue at, euismod arcu. Morbi quis nunc lorem. Curabitur sit amet semper orci. Suspendisse tellus nisi, ullamcorper sed feugiat ac, mollis sed lorem.
 
\subsection{Subsection1 name}

 Curabitur elementum orci non arcu placerat vehicula. Sed commodo urna nec elit ultricies, vitae dignissim felis aliquam. Sed nisi felis, luctus in aliquet vel, consectetur ut quam. Aliquam vel ante ac nunc dignissim pellentesque. Vivamus porta molestie urna.

$$
E = mc^2
$$

\subsection{Subsection2 name}
{\centering
\includesvg[clean, width=\columnwidth]{images/fig}
\caption{Figure name}
\vspace{1ex}
\includesvg[clean, width=0.6\columnwidth]{images/fig}
\vspace{2ex}
\caption{You can scale figures and text separately}

}

%%%%%%%%%%%%%%%%%%%%%%%%%%%%%%%%%%%%%%%%%%%%%%%%%%%%%%%%%%%%%%%%%%%%%%%%%%%%%%
% Second column %%%%%%%%%%%%%%%%%%%%%%%%%%%%%%%%%%%%%%%%%%%%%%%%%%%%%%%%%%%%%%
\newcolumn

\section{Another Section} \justifying
Donec urna dolor, vehicula ut eleifend at, suscipit in ante. Nam ornare vulputate tellus, eget auctor nibh bibendum quis. Ut dictum nisl sed urna ultrices, accumsan varius purus vehicula. Etiam hendrerit turpis eu ex condimentum lacinia. Mauris accumsan neque nec mi pharetra, ultrices placerat magna porttitor. Donec a dolor sollicitudin, vehicula sem eu, cursus ligula. Duis iaculis mi ac placerat ullamcorper. Nunc dignissim sit amet urna non rhoncus.

\vspace{2ex}
\newcommand{\figwidth}{0.7\columnwidth}
\begin{columns}[c]
    \begin{column}{0.95\columnwidth-\figwidth}
        \begin{itemize}   \itemsep15pt    
            \item First item
            \item Second item
        \end{itemize}
        
        $$
          E = mc^2
        $$
        
    \end{column}
    
    \begin{column}{0.005\columnwidth}
    \end{column}
    
    \begin{column}{\figwidth}
        \centering{
            \figfont
            \includesvg[clean, width=\columnwidth]{images/fig}
            \vspace*{1ex}
            \caption{Figure name} 
        } 
    \end{column}
\end{columns}

Nulla eu mi varius, vehicula leo vel, tincidunt dolor. Phasellus vitae iaculis neque, rutrum posuere nunc. Maecenas nec est at est sodales tincidunt sit amet id dui. Nam lacinia pulvinar turpis, at malesuada mi auctor elementum. Maecenas interdum accumsan odio, eu interdum mi consectetur ut. In sed luctus neque, quis posuere ex. Nam vel ultricies eros, sit amet rhoncus ante.

Nullam sollicitudin vitae felis vel accumsan. In hac habitasse platea dictumst. Morbi posuere et est sit amet scelerisque. Fusce sagittis risus luctus massa convallis rhoncus. Aliquam consequat id nibh eu eleifend. Vestibulum laoreet orci placerat ante pulvinar mattis. Aenean non dolor eu nisl pharetra finibus a sit amet urna. Phasellus id accumsan diam. Nam pellentesque eget quam ac egestas. Nullam vestibulum felis sit amet tortor porttitor viverra. Nunc finibus sem et ex aliquet, vitae viverra justo efficitur. Donec luctus velit eu dolor blandit tempor. Cras malesuada nunc quis justo blandit, in convallis leo elementum. Donec accumsan iaculis faucibus. Integer sit amet quam dictum, finibus mi sed, dapibus lacus. Nulla at lacus vel massa tincidunt commodo.

Suspendisse a erat volutpat dolor varius ornare et a justo. Etiam ultrices arcu ex, eget placerat magna lacinia at. Nunc interdum non arcu quis posuere. In hac habitasse platea dictumst. Praesent dictum vehicula sollicitudin. Phasellus orci enim, ultricies at facilisis sit amet, viverra non nulla. Ut eget aliquet risus. Sed vitae sodales ex, quis porttitor nibh. Etiam finibus ipsum a aliquam maximus. Phasellus euismod iaculis turpis eu lobortis. Nullam eu orci auctor, sagittis urna sit amet, ultrices metus. Phasellus et dui vel quam finibus tristique. Cras et risus sed diam porttitor tristique. Etiam bibendum purus eget orci semper, ac rutrum nisi viverra.

\section{Summary} \justifying
\begin{itemize}
    \item First item
    \item second item
    \item Another item
\end{itemize}


\end{poster}
\end{document}